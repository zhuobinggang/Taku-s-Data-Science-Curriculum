\documentclass[a4paper]{article}
\usepackage{xeCJK}
\setCJKmainfont{IPAMincho}
\setCJKsansfont{IPAGothic}
\setCJKmonofont{IPAGothic}
\usepackage{graphicx}
\title{企画書}
\date{2020年10月20日}
\author{卓 秉綱}

\usepackage[
backend=biber,
style=alphabetic,
sorting=ynt
]{biblatex}

\addbibresource{qhe.bib}


\begin{document}
\maketitle
研究テーマ:段落分割による文書のトピック分割を用いて重要文抽出\\

\section{概要}
自動要約の目的は、主旨を保つながら文書を短く圧縮すること。自動要約の手法は、抽出的(extractive)要約と生成的(abstractive)要約に分けられる。本研究は抽出的要約に注目する。\\

従来の抽出手法は、一文ずつ読み取り、その文を抽出すべきか否かを2分類する。その手法は文脈を無視し、独立した文だけを考慮する故に良い結果を出せないと思われる。\\

最近、MingZhongら\cite{zhong2020extractive}は、新たな抽出手法を提案した。MingZhongらは抽出の任務を「意味的テキストマッチング(semantic text matching)」と再定義し、文と文書全体をBertで特徴ベクトルに転換し、その類似度(ユークリッド距離)に基づき、文を抽出するか否かを決める。その研究は最先端の成果を出した。\\

しかし、MingZhongらの研究には一つの問題点がある。それは、一つの文書に複数のトピックが存在することであり(例えば、とあるフランスの紹介文章に、第一部分のトピックはフランスの地理環境、第二部分はフランスの歴史)、それに対して、通常、一文は複数のトピックに所属することができない。それゆえ、その文と文書のマッチングはアンバランスということは明らかである。本研究は、その問題は要約に悪影響をもたらすことを仮定する。\\

そのため、本研究では、抽出的要約の精度を向上させる手法を提案する。まず、文書をトピックに分割し、トピック別でMingZhongらの手法を用いって複数の要約文を抽出する。最後にそれらの要約文を組み合わせて結果を出力する。文書に対してトピックを分割するために、本研究ではニューラルネットワークに基づくトピック分割モデルを提案する。モデルを予備訓練するために、段落分割に基づく自己教師あり(self-supervised)の訓練手法を提案する。予備訓練した後、WikiSection\cite{arnold2019sector}を用いってモデルを微調整する。\\

\section{先行研究}


\subsection{MingZhongらの抽出手法}

MingZhongら\cite{zhong2020extractive}は、まずcontent selection moduleを使用して余分な文を削除し、残りの文を候補要約としてBertで埋め込みを取得、そして文書のユークリッド距離と比較し、結果として最も近い距離を持つ候補要約を出力。

\subsection{トピック分割}

Arnoldら\cite{arnold2019sector}は、まず文をword2vecで特徴ベクトルに転換し、BLSTMを使って両方向の文脈に基づきその文を複数のトピックカテゴリに分類する。\\


\section{問題点と着眼点}

既存の抽出方法のほとんどはトピック分割の問題を考慮しないため、本研究では2段階の抽出方法を提案する。まず、トピックを分割する。次に、トピック別で抽出してから組み合わせて出力する。この方法で抽出の精度の向上を期待する。

\section{有用性}

まず、トピックを分割してから抽出的要約を行うと、抽出の精度の向上が期待できる。\\

次に、提案された訓練手法を通じて、トピック分割モデルを獲得できる。抽出的要約に限らず、トピックを理解したモデルは色んなことに役立てると思う。例えば、対話システムに、ユーザーとの対話が新しいトピックに転換したと判断できれば、古い文脈の破棄ができる、全部の文脈を記憶しなくて済む。インターネットで大量の情報を処理する時にも役立てるかと思う。

\section{新規性}

提案された2段階の抽出方法には新規性があり、段落分割によるトピック分割モデルの予備訓練、その手法にも新規性がある。

\section{実験方法}

トピック分割モデルを提案する。このモデルは、文全体をSentence-Bert\cite{reimers2019sentence}で特徴ベクトルに転換し、ポインタネットワーク\cite{vinyals2015pointer}に入力してから、分割すべきの文の位置を出力する。その結果に基づき、文書のトピック分割は簡単にできる。\\

トピック分割モデルを予備訓練するために、自己教師あり(self-supervised)の学習手法を提案する:コーパスから文書を読み取り、その文書の全ての段落を一つに合わせ、モデルに段落の再分割をさせる。その訓練により、モデルがトピックへの理解を学習できると期待する。\\

予備訓練した後、WikiSectionを用いってモデルを微調整する。\\

\printbibliography


\end{document}
